%%%%%%%%%%%%% local definitions %%%%%%%%%%%%%%%%%%%%%

%\setcounter{section}{-1} % Start from Section 0

% Decays
\newcommand{\bmux}{\ensuremath{b{\to}\mu{X}}\xspace}
\newcommand{\bmx}{\ensuremath{B^\pm{\to}\mu{X}}\xspace}
\newcommand{\bhll}{\ensuremath{B{\to}H\ell\ell}\xspace}
\newcommand{\bhee}{\ensuremath{B{\to}Hee}\xspace}
\newcommand{\bkll}{\ensuremath{B^{\pm}{\to}K^{\pm}\ell\ell}\xspace}
\newcommand{\bkee}{\ensuremath{B^{\pm}{\to}K^{\pm}ee}\xspace}
\newcommand{\bkstll}{\ensuremath{B^{0}{\to}K^{*}\ell\ell}\xspace}
\newcommand{\bkstee}{\ensuremath{B^{0}{\to}K^{*}ee}\xspace}
\newcommand{\bkstmm}{\ensuremath{B^{0}{\to}K^{*}\mu\mu}\xspace}
\newcommand{\bkstjpsimm}{\ensuremath{B^{0}{\to}K^{*}J/\psi(\mu\mu)}\xspace}

% RK related 
\newcommand{\qsq}{\ensuremath{q^2}\xspace}
\newcommand{\rk}{\ensuremath{R_{\text{K}}}\xspace}
\newcommand{\bparking}{$B$ parking\xspace}

% Luminosity related
\newcommand{\linst}{\ensuremath{\mathcal{L}_{\text{inst}}}\xspace}
\newcommand{\lint}{\ensuremath{L_{\text{int}}}\xspace}
\newcommand{\invcmsq}{\ensuremath{\,\text{cm}^{-2}}\xspace}
\newcommand{\invcmsqs}{\ensuremath{\,\text{cm}^{-2}\,\text{s}^{-1}}\xspace}
\newcommand{\fbpm}{\ensuremath{f_{b{\to}B^\pm}}\xspace}
\newcommand{\pb}{\ensuremath{\,\text{pb}}\xspace}

% Kinematic
\newcommand{\ptgen}{\ensuremath{\pt^{\text{gen}}}\xspace}
\newcommand{\etagen}{\ensuremath{\eta^{\text{gen}}}\xspace}
\newcommand{\ipsig}{\ensuremath{\text{IP}_{\text{sig}}}\xspace}

% Acc. times Eff.
\newcommand{\acc}{\ensuremath{\mathcal{A}}\xspace}
\newcommand{\accgen}{\ensuremath{\acc^{\text{gen}}}\xspace}
\newcommand{\accsignal}{\ensuremath{\mathcal{A}_{\text{signal}}}\xspace}
\newcommand{\eff}{\ensuremath{\varepsilon}\xspace}
\newcommand{\effvar}[1]{\ensuremath{\eff^{\text{#1}}}\xspace}
\newcommand{\efflvl}{\ensuremath{\effvar{L1}}\xspace}
\newcommand{\effhlt}{\ensuremath{\effvar{HLT}}\xspace}
\newcommand{\effreco}{\ensuremath{\effvar{reco}}\xspace}
\newcommand{\effnano}{\ensuremath{\effvar{nano}}\xspace}
\newcommand{\effpresel}{\ensuremath{\effvar{pre-sel}}\xspace}
\newcommand{\effbdt}{\ensuremath{\effvar{BDT}}\xspace}
\newcommand{\effqsq}{\ensuremath{\effvar{\qsq}}\xspace}
\newcommand{\axe}{\ensuremath{\acc{\times}\eff}\xspace}

% Misc
\newcommand{\cf}{{\it cf.}\xspace}
\newcommand{\nsig}{\ensuremath{N_{\text{signal}}}\xspace}
\newcommand{\sci}[2]{\ensuremath{#1{\times}10^{#2}}\xspace}
\renewcommand{\bbbar}{\ensuremath{\text{b}\overline{\text{b}}}\xspace}
\newcommand{\nbb}{\ensuremath{N_{\bbbar}}\xspace}
\newcommand{\un}{\textunderscore}
\newcommand{\egamma}{\ensuremath{e/\gamma}\xspace}
\newcommand{\ra}{\ensuremath{\rightarrow}\xspace}


%%%%%%%%%%%%%%%  Title page %%%%%%%%%%%%%%%%%%%%%%%%
\cmsNoteHeader{AN-21-XXX}
% >> Title: please make sure that the non-TeX equivalent is in
% PDFTitle below for papers. For PASs, PDFTitle can be used with plain
% TeX.
\title{Performance of the inclusive di-electron trigger}

% >> Authors Author is always "The CMS Collaboration" for PAS and
% papers, so author, etc, below will be ignored in those cases For
% multiple affiliations, create an address entry for the combination
% To mark authors as primary, use the \author* form

\author[icl]{Robert Bainbridge}
\author[icl]{Jay Odedra}
%\author[zurich]{Yuta Takahashi}
%\author[brown]{Xuli (Sebastian) Yan}

\address[icl]{Imperial College London}
%\address[zurich]{University of Z\"urich}

\date{\today}

% >> Abstract Abstract processing: 1. **DO NOT use \include or
% \input** to include the abstract: our abstract extractor will not
% search through other files than this one.  2. **DO NOT
% use %** to comment out sections of the abstract: the extractor will
% still grab those lines (and they won't be comments any longer!).
% 3. For PASs: **DO NOT use CMS tex macros.**...in the abstract: CDS
% MathJax processor used on the abstract doesn't understand them _and_
% will only look within $$. The abstracts for papers are hand
% formatted so macros are okay.
\abstract{}

% >> PDF Metadata Do not comment out the following hypersetup lines
% (metadata). They will disappear in NODRAFT mode and are needed by
% CDS.  Also: make sure that the values of the metadata items are
% sensible and are in plain text with the possible exception of the
% PDFtitle for a PAS. Then you can use pure TeX symbols as if on a
% typewriter. Examples: $\sqrt{s}=13\TeV$ => $sqrt{s}=$ 13 TeV;
% 32\fbinv => 32 fb$^{-1}$ No unescaped comment % characters.
% No curly braces {} except for TeX in the PDFtitle.
\hypersetup{%
pdfauthor={Robert Bainbridge, Jay Odedra},% 
pdftitle={Performance of the inclusive di-electron trigger},%
pdfsubject={CMS},%
pdfkeywords={CMS}} % limit six total

\maketitle %maketitle comes after all the front information has been supplied
\setcounter{tocdepth}{2}
\tableofcontents 

\clearpage
\section{Introduction}

The detailed proposal for a new di-electron trigger for Run~3 that
targets a precision measurement of \rk (and related observables) can
be found at Ref.~\cite{AN-21-160}. The logic is sufficiently inclusive
to be relevant for any di-electron final state arising from b hadron
(semi)-leptonic decays and over the full \qsq range, as well as any
low-mass resonance (below 6\GeV) that is centrally produced. Simple
kinematic requirements are employed by the logic, which are tuned to
maximise the number of signal-like \bkee candidates at low \qsq while
satisfying all constraints from the trigger and DAQ systems. The
anticipated performance of the di-electron trigger for 2022 and Run~3
is expected to yield in excess of 500 and 1000 signal candidates,
respectively, from an analysis of the recorded data sample. For a
point of reference: the number of \bkee signal candidates at low \qsq
obtained from an analysis of the \bparking data sample recorded in
2018 is expected to be approximately 15; and the LHCb experiment
reported 1640 candidates in their analysis of the full Run~2
data-taking period~\cite{LHCb:2021trn}.

This document outlines the studies performed to commission the trigger
logic, using data recorded early in 2022, and to determine the physics
performance of the trigger.

\subsection{Key commissioning tasks}

We list below a list of key tasks related to the commissioning the
di-electron trigger logic, the menu strategy, the use of control
triggers, and the physics performance.

Several tasks (\eg trigger rate, timing, purity measurement) are
crucial from the perspective of demonstrating the efficient use of CMS
resources, as well as providing guidance on potential interventions
during 2022 and on how to improve the trigger performance for 2023 and
beyond. A subset of the tasks will help to quantify the physics
performance: a study of the trigger efficiency turn-on as a function
of kinematical variables relevant to the analysis, and the subsequent
determination of data/MC scale factors; and an estimate of the number
of candidates recorded at low \qsq using early data. The latter task
will permit a crosscheck of the per-path signal \axe estimates
provided by the di-electron trigger proposal~\cite{AN-21-160}, which
in turn will help to solidify the end-of-year estimates provided
therein.

The outcome of all commissioning studies and any required
actions/interventtions should be reported to the BPH trigger working
group and TSG.

\subsubsection{Trigger rates} 

\begin{itemize}
\item This concerns the determination of the di-electron trigger rate
  at both the L1 and HLT.
\item Rates should be measured and monitored from the first instance
  of the di-electron trigger being included in the menu. Estimates of
  the L1 rate behaviour for the CMS menu as a whole have large
  uncertainties. Prompt intervention may be required to regulate rates
  within the available L1 budget. This is especially important for the
  di-electron trigger during early operations, as the L1 rate will be
  tens of kHz given that the stategy is to utilise all available spare
  capacity at low values of \linst.
\item Rates should be monitored for each of the ${\sim}10$ L1 seeds
  and HLT paths. Each HLT path is seeded by the logical OR of all
  available L1 seeds to provide maximum flexibility in controlling
  rate while maximising signal \axe. The L1 seeds and HLT paths are
  pairwise selected by unity prescale values; all other pairs are
  effectively removed from the menu by selecting zero prescale
  values. 
\item In the case of anomalous rate beahviour, backup paths should be
  selected according to a predefined strategy, as outlined in
  Sec.~\ref{}.
\item The rate dependence on pileup should be determined per L1 seed
  and HLT path.
\end{itemize}

\subsubsection{CPU timing studies for the HLT}

\begin{itemize}
\item 
\end{itemize}

This concerns studies to evaluate the

\subsubsection{Measure trigger efficiencies}

\begin{itemize}
\item 
\end{itemize}

Mainly "service work for CMS" and to satisfy TSG. Can provide turn ons
using (up to) three control triggers (as discussed). Lots of paths to
study, which is a pain. The most relevant aspects for the analysis is
ensuring we understand the turn-on if operating there, and determining
data/MC scale factors. The di-electron trigger has no displacement
requirement and so we won't suffer from a slow turn-on (vs dxy) like
the 2018 muon trigger. Main kine vars are pT and eta, although we have
a deltaR req at L1 and di-electron mass req at HLT. Perhaps mass
resolution might be poor, but we cut at 6\GeV. Surely loose enough to
cover full \qsq range, but check for slow turn-on at high \qsq? We
already know (muon) trigger eff has little effect on RK and is a
subdominant syst, but is this also valid for di-electron? We'll start
with SingleMuon control trigger. We'll analyse 2018 BParking data set
to have prelim estimates before 2022 data taking.

\subsubsection{Measure trigger purity}

\begin{itemize}
\item 
\end{itemize}

Mainly "service work for CMS" and to satisfy TSG. Extract signal
counts on J/psi peak and use MC to unfold Acc x Eff x BF, and compare
counts to number of di-ele triggers fired. Mainly for TSG.

\subsubsection{Validation of physics performance}



\begin{itemize}
\item 
\end{itemize}

The projections for number of expected candidates at low \qsq. As
determined in Ref.~\cite{AN-21-160}. We need to extract signal counts
on J/psi peak and extrapolate to low q2 using MC. Perfect for Jay. I
will help.

\clearpage
\subsection{Control trigger for efficiency measurements}

\subsubsection{Single muon}

\begin{align}
  N^{\text{produced}} & = \lint \times \sigma_{\bbbar} \times
  f_{b_{1}} \times f_{b_{2}} \times \mathcal{B}[b_1{\to}{\mu}X] \times
  \mathcal{B}[b_2{\to}{J}{/}{\psi}({\to}ee)X] \times 2 &
  \label{equ:mu-produced}
  \\[0.5ex]
  N^{\text{recorded}} & = N^{\text{produced}} \times
  {\acc}[b_1{\to}{\mu}X] \times \effvar{trigger}_{b_1{\to}{\mu}X} &
  \text{"tag-side"}
  \label{equ:mu-recorded}
  \\[0.5ex]
  N^{\text{tag}} & = N^{\text{recorded}} \times
  \effvar{analysis}_{b_1{\to}{\mu}X} & \text{"tag-side"}
  \label{equ:mu-tag}
  \\[0.5ex]
  N^{\text{signal}} & = N^{\text{tag}} \times
  {\acc}[b_2{\to}{J}{/}{\psi}({\to}ee)X] \times
  \effvar{analysis}_{b_2{\to}{J}{/}{\psi}({\to}ee)X} &
  \text{"signal-side"}
  \label{equ:mu-signal}
\end{align}

\begin{table}[!th]
  \centering
  \caption{}
  \label{tab:mu-production}
  \footnotesize
  \def\arraystretch{1.2}
  \newcommand{\trigaxe}{\ensuremath{{\acc}[b_1{\to}{\mu}X] \times \effvar{trigger}_{b_1{\to}{\mu}X}}\xspace}
  \begin{tabular}{lll}
    \hline
                                           & Variable                                         & Value               \\
    \hline
    \multicolumn{3}{l}{\it Production:}                                                                             \\
    Integrated luminosity                  & \lint                                            & 1.0\fbinv           \\
    \bbbar cross section (13\TeV)          & $\sigma_{\bbbar}$
                                           & $\sci{4.7}{11}$~fb                                                     \\ 
    Fragmentation fractions                & $f_{b_1}, f_{b_2}$                                   & 1.0 \\
    Branching fraction (tag-side)          & $\mathcal{B}[b_1^\pm{\to}{\mu}X]$                & 0.2                 \\
    Branching fraction (signal-side)       & $\mathcal{B}[b_2{\to}{J}{/}{\psi}\,X]$               & 0.01                \\
    Branching fraction (signal-side)       & $\mathcal{B}[{J}{/}{\psi}{\to}ee]$                   & 0.06                \\
    \multicolumn{3}{l}{\it Tag-side:}                                                                               \\
    Trigger (\texttt{HLT\_Mu9\_IP6})       & \trigaxe                                         & $\sci{1.25}{-3}$    \\
    Selection of ${\mu}X$ events           & $\effvar{analysis}_{b_1{\to}{\mu}X}$             & 0.8 (?)             \\ 
    \multicolumn{3}{l}{\it Signal-side analysis:}                                                                   \\
    Acceptance ($\pt > 2\GeV, |\eta|<1.2$) & ${\acc}[b_2{\to}{J}{/}{\psi}({\to}ee)X]$             & 0.7 (?)             \\
    Selection of $eeX$ events              & $\effvar{analysis}_{b_2{\to}{J}{/}{\psi}({\to}ee)X}$ & 0.02 (?)            \\ 
    \hline
  \end{tabular}
\end{table}

Given the values in Table~\ref{tab:mu-production}, and assuming that
we consider inclusively the decays from all b hadrons (\eg $B^\pm$,
$B^0$, $B_{\text{c}}$, $\Lambda_{\text{b}}$, \dots) such that $f_b =
1.0$, then the number of \bbbar events produced at the LHC that decay
to $b_1^\pm{\to}{\mu}X, b_2{\to}{J}{/}{\psi}({\to}ee)X$ is determined
to be $N^{\text{produced}} = \sci{1.1}{8}$ per \fbinv. The number of
$b{\to}{\mu}X$ decays that are recorded by the trigger is determined
to be $N^{\text{recorded}} = \sci{1.4}{5}$ per \fbinv. The number of
$b{\to}{\mu}X$ decays that are reconstructed offline is determined to
be $N^{\text{tag}} = \sci{1.1}{5}$ per \fbinv. The number of
signal-side $eeX$ events that are reconstructed offline is determined
to be $N^{\text{signal}} = 1600$ per \fbinv.

\begin{table}[!th]
  \centering
  \caption{Estimated \bkee candidates accumulated for an assumed rate
    allocation of 3.5~kHz for the di-electron trigger, for different rate
    allocations {\it for the single muon trigger}, and as a function of
    luminosity profiles that are representative of how the LHC
    performance might evolve during 2022. } 
  \label{tab:estimates-with-single-mu}
  \footnotesize
  \begin{tabular}{ccccccccc}
    \hline
    Peak \linst & \lint (per fill) & $N_{\textrm{fill}}$ & \lint ($N$ fills) & \multicolumn{4}{c}{Candidates (per allocation)} \\
    \cline{5-9}
    $[10^{34}\invcmsqs]$ & [$\fbinv$] &  & [$\fbinv$] & 0~kHz & 5~kHz & 10~kHz & 15~kHz & 18.3~kHz \\
    \hline
    \multicolumn{4}{l}{\it Intensity ramp-up:} \\
    0.24 & 0.09 & 3  & 0.28 & 9.5   &   9.5 &   9.5 &   9.5 &   9.5 \\
    0.47 & 0.18 & 3  & 0.54 & 23.7  &  15.9 &  15.9 &  15.9 &  15.9 \\
    0.70 & 0.27 & 3  & 0.81 & 15.9  &  22.6 &  22.6 &  22.6 &  22.6 \\
    \multicolumn{4}{l}{\it pp collisions for physics:} \\
    0.9  & 0.35 & 12 & 4.17 & 102.1 & 101.0 & 101.0 &  87.9 &  87.9 \\
    1.1  & 0.42 & 12 & 5.10 & 105.0 & 104.8 & 104.8 & 100.7 &  88.5 \\
    1.3  & 0.50 & 12 & 6.02 & 105.8 & 105.8 & 105.8 &  85.4 &  82.1 \\
    1.5  & 0.58 & 12 & 6.95 & 96.5  &  96.5 &  89.6 &  85.3 &  62.5 \\
    1.7  & 0.66 & 12 & 7.88 & 73.8  &  73.8 &  64.6 &  48.8 &  37.4 \\
    2.0  & 0.78 & 12 & 9.36 & 39.1  &  39.1 &  35.7 &  32.9 &  26.8 \\
    \multicolumn{4}{l}{\it Totals:} \\
    -    & -    & 81 & 41.1 & 571.3 & 568.9 & 549.5 & 489.1 & 433.2 \\
    \hline
  \end{tabular}
\end{table}



\newpage
\begin{acknowledgments}
% RK analyzers? 
\end{acknowledgments}

\newpage
\bibliography{AN-21-XXX}

%\appendix
%\numberwithin{figure}{section}
%\numberwithin{table}{section}
%\input{sections/appendix}

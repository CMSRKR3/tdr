\clearpage
\section{Trigger performance}
\label{sec:trigger}

\subsection{Introduction}

\fixme{AUTHOR?}

%Can provide turn ons using (up to) three control triggers (as
%discussed). Multiple paths, which complicates matters. The most
%important aspect for the analysis is to ensure we understand the
%turn-on if operating there, and determining data/MC scale factors. The
%di-electron trigger has no displacement requirement and so we won't
%suffer from a slow turn-on (vs dxy) like the 2018 muon trigger. Main
%kine vars are pT and eta, although we have a deltaR req at L1 and
%di-electron mass req at HLT (upper bound of 6\GeV). We already know
%(muon) trigger eff has little effect on RK(mu) and is a subdominant
%syst, but is this also valid for di-electron?

\subsection{Identifying triggers used to record data}

\fixme{Rob, Jay?}

\fixme{TriggerLuminosity package. Path and lumi table. Etc}

\subsection{Efficiency studies using a tag-side single-muon control trigger}

\fixme{AUTHOR?}

\fixme{Control trigger logic. Data sample. Offline
  selection. Method. Results. Etc}

%  BELOW IS LIKELY OUT OF DATE GIVEN CHANGES TO PT AND ETA THRESHOLDS AND PRESCALES
%\begin{align}
%  N^{\text{produced}} & = \lint \times \sigma_{\bbbar} \times
%  f_{b_{1}} \times f_{b_{2}} \times \mathcal{B}[b_1{\to}{\mu}X] \times
%  \mathcal{B}[b_2{\to}{J}{/}{\psi}({\to}ee)X] \times 2 &
%  \label{equ:mu-produced}
%  \\[0.5ex]
%  N^{\text{recorded}} & = N^{\text{produced}} \times
%  {\acc}[b_1{\to}{\mu}X] \times \effvar{trigger}_{b_1{\to}{\mu}X} &
%  \text{"tag-side"}
%  \label{equ:mu-recorded}
%  \\[0.5ex]
%  N^{\text{tag}} & = N^{\text{recorded}} \times
%  \effvar{analysis}_{b_1{\to}{\mu}X} & \text{"tag-side"}
%  \label{equ:mu-tag}
%  \\[0.5ex]
%  N^{\text{signal}} & = N^{\text{tag}} \times
%  {\acc}[b_2{\to}{J}{/}{\psi}({\to}ee)X] \times
%  \effvar{analysis}_{b_2{\to}{J}{/}{\psi}({\to}ee)X} &
%  \text{"signal-side"}
%  \label{equ:mu-signal}
%\end{align}
%
%\begin{table}[!th]
%  \centering
%  \caption{}
%  \label{tab:mu-production}
%  \footnotesize
%  \def\arraystretch{1.2}
%  \newcommand{\trigaxe}{\ensuremath{{\acc}[b_1{\to}{\mu}X] \times \effvar{trigger}_{b_1{\to}{\mu}X}}\xspace}
%  \begin{tabular}{lll}
%    \hline
%                                           & Variable                                         & Value               \\
%    \hline
%    \multicolumn{3}{l}{\it Production:}                                                                             \\
%    Integrated luminosity                  & \lint                                            & 1.0\fbinv           \\
%    \bbbar cross section (13\TeV)          & $\sigma_{\bbbar}$
%                                           & $\sci{4.7}{11}$~fb                                                     \\ 
%    Fragmentation fractions                & $f_{b_1}, f_{b_2}$                                   & 1.0 \\
%    Branching fraction (tag-side)          & $\mathcal{B}[b_1^\pm{\to}{\mu}X]$                & 0.2                 \\
%    Branching fraction (signal-side)       & $\mathcal{B}[b_2{\to}{J}{/}{\psi}\,X]$               & 0.01                \\
%    Branching fraction (signal-side)       & $\mathcal{B}[{J}{/}{\psi}{\to}ee]$                   & 0.06                \\
%    \multicolumn{3}{l}{\it Tag-side:}                                                                               \\
%    Trigger (\texttt{HLT\_Mu9\_IP6})       & \trigaxe                                         & $\sci{1.25}{-3}$    \\
%    Selection of ${\mu}X$ events           & $\effvar{analysis}_{b_1{\to}{\mu}X}$             & 0.8 (?)             \\ 
%    \multicolumn{3}{l}{\it Signal-side analysis:}                                                                   \\
%    Acceptance ($\pt > 2\GeV, |\eta|<1.2$) & ${\acc}[b_2{\to}{J}{/}{\psi}({\to}ee)X]$             & 0.7 (?)             \\
%    Selection of $eeX$ events              & $\effvar{analysis}_{b_2{\to}{J}{/}{\psi}({\to}ee)X}$ & 0.02 (?)            \\ 
%    \hline
%  \end{tabular}
%\end{table}
%
%Given the values in Table~\ref{tab:mu-production}, and assuming that
%we consider inclusively the decays from all b hadrons (\eg $B^\pm$,
%$B^0$, $B_{\text{c}}$, $\Lambda_{\text{b}}$, \dots) such that $f_b =
%1.0$, then the number of \bbbar events produced at the LHC that decay
%to $b_1^\pm{\to}{\mu}X, b_2{\to}{J}{/}{\psi}({\to}ee)X$ is determined
%to be $N^{\text{produced}} = \sci{1.1}{8}$ per \fbinv. The number of
%$b{\to}{\mu}X$ decays that are recorded by the trigger is determined
%to be $N^{\text{recorded}} = \sci{1.4}{5}$ per \fbinv. The number of
%$b{\to}{\mu}X$ decays that are reconstructed offline is determined to
%be $N^{\text{tag}} = \sci{1.1}{5}$ per \fbinv. The number of
%signal-side $eeX$ events that are reconstructed offline is determined
%to be $N^{\text{signal}} = 1600$ per \fbinv.

\subsection{Efficiency studies using tag-side di-muon control triggers}

\fixme{Noah, GK?}

\fixme{Control trigger logic. Data sample. Offline
  selection. Method. Results. Etc}

\subsection{Efficiency studies using single-electron control triggers}

\fixme{Claudio, Chiara?}

\fixme{Control trigger logic. Data sample. Offline
  selection. Method. Results. Etc}

\subsection{Final trigger efficiency scale factors}

\fixme{AUTHOR?}

\fixme{Should cover consistency checks between results from the
  different control triggers. Final scale factors and related
  systematics.}

\subsection{Trigger purity}

\fixme{AUTHOR?}

\fixme{Do we try to estimate the trigger purity? Definition of purity?
  Fraction of triggered events that record $b\bar{b}$ with at least
  one b hadron decay to the $eeX$ final state? e.g. Extract signal
  counts on J/psi peak and use MC to unfold Acc x Eff x BF, and
  compare counts to number of di-ele triggers fired?}

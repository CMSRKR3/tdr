\clearpage
\section{Introduction}

The detailed proposal for a new di-electron trigger for Run~3 that
targets a precision measurement of \rk (and related observables) can
be found at Ref.~\cite{AN-21-160}.

The trigger logic is sufficiently inclusive to be relevant for any
di-electron final state arising from b hadron (semi)-leptonic decays
over the full \qsq range. Furthermore, the logic may be of interest to
other PAGs that pursue final states containing at least two low-\pt
electrons that are produced centrally, promptly or otherwise, with at
least one electron-pair satisfying $m(ee) < 6\GeV$.

Simple kinematic requirements are employed by the trigger logic, which
are tuned to maximise the number of signal-like \bkee candidates at
low \qsq while satisfying all constraints from the trigger and
computing systems. The di-electron trigger for 2022 was expected to
yield a large number of \bkee candidates at low \qsq under "ideal"
operating conditions from the LHC and CMS (as detailed in
Ref.~\cite{AN-21-160}).
%Up to $\sim$500 candidates were possible. As points of reference: the
%number of \bkee signal candidates at low \qsq obtained from an
%analysis of the \bparking data sample recorded in 2018 is expected to
%be approximately 15; and the LHCb experiment reported 1640 candidates
%from their analysis of 9\fbinv of data recorded during the Run~1 and
%Run~2 data-taking periods~\cite{LHCb:2021trn}.

This document the details the integration of the di-electron triggers
during the 2022 data-taking period and their operational
parameters. Further, several studies of the data set recorded by the
triggers during 2022 are provided:
\begin{itemize}
\item Multiple trigger paths are used to record the data set. The
  intervals for which each trigger path was enabled, as well as the
  corresponding integrated luminosities recorded by each path, are
  identified.
\item Several studies quantify the trigger logic performance in terms
  of the efficiency to record the $eeX$ final state as a function of
  several kimematic variables. The accuracy of the (MC-based)
  modelling of these efficiency "turn-on curves" was also quantified,
  in terms of data-to-MC scale factors.
\item A (blind) analysis was performed to an estimate of the number of
  \bkee candidates at low \qsq that were recorded during 2022 and
  satisfy all analysis-level selections.
\end{itemize}
All of these studies have been reported to the BPH trigger working
group and the Trigger Studies Group. Finally, a proposal is made to
maintain the same trigger strategy during the 2023 data-taking period.

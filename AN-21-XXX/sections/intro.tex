\clearpage
\section{Introduction}

The detailed proposal for a new di-electron trigger for Run~3 that
targets a precision measurement of \rk (and related observables) can
be found at Ref.~\cite{AN-21-160}. The logic is sufficiently inclusive
to be relevant for any di-electron final state arising from b hadron
(semi)-leptonic decays and over the full \qsq range, as well as any
low-mass resonance (below 6\GeV) that is centrally produced. Simple
kinematic requirements are employed by the logic, which are tuned to
maximise the number of signal-like \bkee candidates at low \qsq while
satisfying all constraints from the trigger and DAQ systems. The
anticipated performance of the di-electron trigger for 2022 and Run~3
is expected to yield in excess of 500 and 1000 signal candidates,
respectively, from an analysis of the recorded data sample. For a
point of reference: the number of \bkee signal candidates at low \qsq
obtained from an analysis of the \bparking data sample recorded in
2018 is expected to be approximately 15; and the LHCb experiment
reported 1640 candidates in their analysis of the full Run~2
data-taking period~\cite{LHCb:2021trn}.

This document outlines the studies performed to commission the trigger
logic, using data recorded early in 2022, and to determine the physics
performance of the trigger.

\clearpage
\section{Data samples and simulation}

\clearpage
\section{Logic}

\subsection{Integration}

JIRA tickets.


\subsection{Menu}


\clearpage
\section{Commissioning}


\subsection{Emulation}

\subsection{Early data}


\clearpage
\section{Performance}



\clearpage
\section{Summary}


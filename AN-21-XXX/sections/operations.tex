\clearpage
\section{Trigger integration and operations during 2022}
\label{sec:operations}

\fixme{AUTHORS? Rob? Sebastian?}
  
\fixme{Menu integration. JIRA tickets. Changes to menu during the
  year. Monitoring L1 and HLT trigger rates. Data streams, data sets,
  prompt and delayed reconstruction. Default and backup paths. Control
  triggers. L1 and HLT purity.}

The section tersely summarises, for the record, some of the important
aspects of integration and operations for the di-electron trigger
paths during the data-taking period of 2022.

\subsection{JIRA tickets}

The following JIRA tickets (with status "Closed" unless stated
otherwise) document the technical exchanges with trigger experts that
concern the implementation of the di-electron trigger paths (and
related control triggers) during 2022.

\begin{itemize}
\item
  \href{https://its.cern.ch/jira/browse/CMSLITDPG-957}{CMSLITDPG-957},
  "New DoubleEG L1 seeds for the B-parking", 08 Nov 2021, initial
  proposal for new DoubleEG L1 seeds, with a link to a presentation
  from Yuta making the physics case during the TSG trigger review on
  22 Jul 2022.
\item
  \href{https://its.cern.ch/jira/browse/CMSLITDPG-997}{CMSLITDPG-997},
  "Activate SingleMu seed for the b-parking in Run-3", 30 Jun 2022,
  activation of \texttt{L1\_SingleMu12er1p5} seed from 1.7E34.
\item \href{https://its.cern.ch/jira/browse/CMSHLT-2273}{CMSHLT-2273},
  "Di-electron triggers for the R(X)-type analysis in Run-3", 12 Apr
  2022. Definition of initial HLT paths and iterations in response to
  CMS problems (\eg HCAL dead time) and LHC partially filled orbits.
\item \href{https://its.cern.ch/jira/browse/CMSHLT-2362}{CMSHLT-2362},
  "SingleMu trigger for B-parking in Run-3", 28 Jun 2022. Definition
  of single-muon control path (\texttt{HLT\_Mu12\_IP6}) to be used
  from 1.7E34 onwards.
\item \href{https://its.cern.ch/jira/browse/CMSHLT-2468}{CMSHLT-2468},
  "Ps columns for BPH needs", 09 Sep 2022. Final prescale column
  definition (in range 0.6-2.2E34) for di-electron triggers.
\item \href{https://its.cern.ch/jira/browse/CMSHLT-2486}{CMSHLT-2486},
  "New Di-electron triggers with dz requirement for the R(X)-type
  analysis in Run-3", 28 Sep 2022. Development paths (deployed during
  late-2022 data-taking) that add the trkHits10 or dz0p8 requirements
  to the nominal paths.
\item \href{https://its.cern.ch/jira/browse/CMSHLT-2487}{CMSHLT-2487},
  "New SingleEle path for the eff. measurement of DoubleEle paths
  targeting R(X) type analyses", 28 Sep 2022. Integration of new
  single-electron control trigger paths (\texttt{HLT\_SingleEle8\_*}).
\item \href{https://its.cern.ch/jira/browse/CMSHLT-2508}{CMSHLT-2508},
  "Deploy of \texttt{HLT\_SingleEle8\_*} online", 13 Oct
  2022. Deployment of single-electron control trigger paths.
\end{itemize}
  
\subsection{Evolution of di-electron trigger menu}

Table~\ref{tab:config} shows the nominal "di-electron trigger menu",
defined in terms of trigger paths with \pt thresholds that evolve as a
function of \linst, as found in the original
proposal~\cite{AN-21-160}.

\begin{table}[!th]
  \centering
  \caption{Di-electron trigger "menus" defined in terms of optimal
    Level-1 and HLT \pt thresholds (and trigger rates) as a function
    of \linst, assuming a nonzero "dedicated" Level-1 trigger rate
    allocation (of 3.5~kHz). Also stated are: the expected acceptance
    times efficiency (\axe); and the \lint and estimated \bkee
    candidates accumulated within an LHC fill of fixed 12-hour
    duration and {\it constant }\linst. Taken from
    Ref.~\cite{AN-21-160}.} 
  \label{tab:config}
  \footnotesize
  \def\arraystretch{1.2}
  \begin{tabular}{ccccccccc}
    \hline
    \linst & \multicolumn{2}{c}{\pt threshold $[\GeVns]$} & &
    \multicolumn{2}{c}{Trigger rate [kHz]} & \axe & \lint [{\fbinv}] &
    Candidates \\
    \cline{2-3}\cline{5-6}\cline{8-9}
    $[10^{34}\invcmsqs]$ & Level-1 & HLT & & Level-1 & HLT &
    $[10^{-4}]$ & \multicolumn{2}{c}{per 12-hour fill} \\
    \hline
    2.0 & 10.5 & 6.5 &  & {\bf 3.5} & 0.30 & 0.13 & 0.86 & 1.88 \\
    1.7 & 8.0  & 5.0 &  & 14.8 & 0.99 & 0.41 & 0.73 & 5.03 \\
    1.5 & 7.0  & 5.0 &  & 28.1 & 1.00 & 0.72 & 0.65 & 7.93 \\
    1.3 & 6.5  & 4.5 &  & 35.2 & 0.88 & 0.97 & 0.56 & 9.18 \\
    1.1 & 6.0  & 4.0 &  & 41.2 & 0.64 & 1.15 & 0.48 & 9.27 \\
    0.9 & 5.5  & 4.0 &  & 51.2 & 0.39 & 1.38 & 0.39 & 9.08 \\
    0.6 & 4.5  & 4.0 &  & 63.0 & 0.09 & 1.73 & 0.26 & 7.57 \\
    \hline
  \end{tabular}
\end{table}

\subsection{Trigger rates}
\label{sec:trigger_rates}

Estimates of the L1 rate behaviour for the CMS menu prior to the start
of data taking in 2022 have large uncertainties. Further, the
di-electron trigger paths are new to the CMS menu. Hence, it is
imperative that the di-electron trigger rates are closely monitored
from the very first instance of the di-electron L1 seeds and HLT paths
being included in the CMS trigger menu. Prompt intervention may be
required to regulate rates within the available L1 budget. This is
especially important for the di-electron trigger paths during early
operations, as the L1 rate will be tens of kHz given that the stategy
is to utilise all available spare capacity at low values of \linst.

Rates should be monitored for each of the ${\sim}10$ L1 seeds and HLT
paths. Each HLT path is seeded by the logical OR of all available L1
seeds to provide maximum flexibility in controlling rate while
maximising signal \axe. The L1 seeds and HLT paths are pairwise
selected by unity prescale values; all other pairs are effectively
removed from the menu by selecting zero prescale values.

In the case of anomalous rate beahviour, backup paths should be
selected according to a predefined strategy, as outlined in
Sec.~\ref{sec:menu}.

The rate dependence on pileup should be determined per L1 seed and HLT
path.

\subsubsection{L1 trigger rates} 

\subsubsection{HLT trigger rates} 

\subsection{Data streams, data sets, reconstruction}

\subsection{Control triggers}
